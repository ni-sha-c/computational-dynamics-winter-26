\documentclass[12pt]{article}
\usepackage[tagged, highstructure]{accessibility}
\usepackage[english]{babel}
\usepackage[utf8x]{inputenc}
\usepackage[T1]{fontenc}
\usepackage[margin=1in]{geometry}
\usepackage{scribe}
\usepackage{listings}
\usepackage{natbib,verbatim}
\usepackage{hyperref}
\hypersetup{
    colorlinks=true,
    linkcolor=blue,
    filecolor=magenta,      
    urlcolor=magenta,
    pdftitle={Course Syllabus},
    pdfauthor={Nisha Chandramoorthy},
    pdflang={en-US}
}

%\Scribe{Your Name}
\title{Syllabus-31310}
\Lecturer{Nisha Chandramoorthy (nishac@uchicago.edu)}
\LectureNumber{\begin{large} Foundations of Computational Dynamics\end{large}}
\LectureDate{}
\LectureTitle{\begin{large}CAAM/STAT 31310\end{large}}

\lstset{style=mystyle}

\begin{document}
\MakeScribeTop

Dynamical systems, often represented as ordinary differential equations (ODEs) or iterations of functions, are models for variables that change with time.
Dynamical systems theory is the study of the structure of all possible solutions of such systems, called orbits or trajectories. When the dynamics is stochastic or deterministic but chaotic, it also makes sense to study the time-asymptotic behavior or \emph{ergodic} properties of orbits. Across science and engineering, several computational questions are asked of dynamical systems such as: i) how sensitive are the orbits and orbit structure of the dynamics to one-time perturbations, to persistent deterministic perturbations, and to noisy perturbations?
ii) given some measurements or observations of the dynamics, how do we forecast orbits into the future or predict their statistical moments?, and iii) how can we construct simpler dynamical systems (with fewer variables) that captures their \emph{essential} features? The primary goals of this course are to put such questions in a mathematical language and study rigorous numerical methods for answering them.

We will start with an overview of nonlinear deterministic systems and their numerical analysis, using dynamical systems and ergodic theory. We will then cover some stochastic analysis and methods from computational statistics. See section \ref{sec:topics} for a list of topics. After establishing mathematical foundations, we will discuss rigorous computations applicable to dynamical systems that arise in the geosciences and in machine learning (optimization algorithms).

\section{General information}
\begin{itemize}
	\item 2 80-minute lectures per week, 3 homeworks, 2 in-class quizzes, 1 final project.
	\item Class time and location: Tuesdays and Thursdays, 12 pm -- 1:20 pm, Kent 103
	\item Class dates: Jan 6, 2026 -- Mar 5, 2026.
	\item Office hours: 3 pm -- 4 pm on Thursdays
	\item Instructor email: nishac@uchicago.edu
\end{itemize}


\section{Resources (not exhaustive)}
\label{sec:resources}
Being interdisciplinary, this course will cover select content from multiple textbooks (and also research articles, which will be cited during class), some of which are listed below. 
\begin{itemize}
	\item 
		\href{https://books.google.com/books?hl=en&lr=&id=9nL7ZX8Djp4C&oi=fnd&pg=PR7&dq=katok+and+hasselblatt&ots=oWieU2cCAH&sig=T-g4-msNhCCTTpoSyMb09ZhnF2Y#v=onepage&q=katok%20and%20hasselblatt&f=false}{Introduction to the modern theory of dynamical systems} by Katok and Hasselblatt, Cambridge University Press.
	\item  \href{https://books.google.com/books?hl=en&lr=&id=wUBvDwAAQBAJ&oi=fnd&pg=PT7&ots=AOtaTsiIyX&sig=MexxyAiFbXhsxZNKq6V7YLjjv5w#v=onepage&q&f=false}{Nonlinear dynamics and chaos} by Strogatz, second edition, Westview Press.
	\item \href{https://link.springer.com/book/10.1007/978-3-662-12616-5}{Numerical Solution of Stochastic Differential Equations} by Kloeden and Platen, Springer. 

	\item \href{https://www.nowpublishers.com/article/Details/MAL-073}{Computational Optimal Transport} by Peyre and Cuturi, Foundations and Trends in Machine Learning, 2019.
	\item \href{https://books.google.com/books?hl=en&lr=&id=dWB9DwAAQBAJ&oi=fnd&pg=PR5&dq=foundations+of+machine+learning+mohri&ots=AznTXOq_s4&sig=oFBecq2rS2nusMY-xRj1qD-0Dsk#v=onepage&q=foundations%20of%20machine%20learning%20mohri&f=false}{Foundations of machine learning} by Mohri, Rostamizadeh and Talwalkar, second edition, MIT Press.
	\item Joel A. Tropp, CMS/ACM 117: Probability Theory \& Computational Mathematics,
Caltech CMS Lecture Notes 2023-01, Pasadena, December 2023. \href{https://doi.org/10.7907/q75sz-e1e79}{Link to notes}
	
\end{itemize}
Lecture notes and handouts will be posted on the \href{https://github.com/ni-sha-c//tree/main}{Github site} as well as the Canvas site for the class.

\section{Grading information and late policy}

This is a discussion-oriented advanced topics course with no exams. The grade will be determined by a final project (40\%) and homeworks (60\%). There will be 5 homeworks, 4 of which will be counted toward the grade (the lowest score will be dropped). The final project will be a research project on a topic of your choice, related to the course material. I will assist you in the selection of a project and designing its scope, if needed. The final project will be due on the last day of class. \\


\textbf{Final project}: The final project has to be done individually, and the deliverables include a proposal, code, accompanying report and a 10-minute in-class presentation. A final project rubric and a set of guidelines will be posted on canvas before the proposal due date. All written material should be typed up and submitted on Gradescope.\\

\textbf{Homeworks}: there will be 5 homework assignments (due dates on Canvas, spread out evenly through the quarter before the final project) that will be theoretical, and often require numerical solutions. You are welcome to discuss with other students and use online resources, including AI assistants such as ChatGPT and Github CoPilot, to solve the questions. After that, however, all the submitted work should be your own. Please submit LaTeX-ed homework solutions (handwritten solutions are often illegible and will not be graded) on Gradescope as a pdf. \\


\textbf{Late policy}: there is a late penalty of 25\% for a submission late by up to 24 hours, 50\% for a submission delayed beyond 24 hours and up to 48 hours. The policy is in place not to punish late submission but 
encourage on-time work and moving on to the next homework. Since I will be considering your best 4 out of 5 homeworks, do not worry about getting late on one of them. Moreover, in case of medical emergencies or other unexpected circumstances, please inform me as soon as you can to prevent the application of the late policy. 

\section{List of topics}
\label{sec:topics}
Please note that the plan below is subject to change, both in terms of the content and order. We will introduce mathematical analyses of dynamical systems through computation and concrete examples, both low-dimensional and high-dimensional.
The syllabus section on Canvas will have a lecture-wise breakdown of topics and is more dynamic. 

\section{Accommodations for Students with Disabilities} 

If you are a student with learning needs that require special accommodation, contact the \href{https://disabilities.uchicago.edu}{Student Disability Services} (SDS). 
Please meet with me to discuss your access needs in this class after you have completed the SDS procedures for requesting accommodations.
SDS contact information: \\
Phone: 773-702-6000 \\
Email: disabilities@uchicago.edu. \\
Website: https://disabilities.uchicago.edu 

%\bibliographystyle{abbrv}           % if you need a bibliography
%\bibliography{mybib}                % assuming yours is named mybib.bib


%%%%%%%%%%% end of doc
\end{document}
